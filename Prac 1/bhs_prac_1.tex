\documentclass[12pt, a4paper]{article}
\usepackage{amsmath}

\title{Control Systems Practical 1}
\date{21 August 2018}
\author{Q. Kruger, 216008466 \and R. de Bruyn, 216969696}

\begin{document}
	\pagenumbering{gobble}
	\maketitle
	\newpage
	\pagenumbering{roman}
	\tableofcontents
	\newpage
	\pagenumbering{arabic}
	
	\section{Introduction} % (fold)
	\label{sec:introduction}
		
	% section introduction (end)

	\section{Equipment and Tools} % (fold)
	\label{sec:equipment_and_tools}
		
	% section equipment_and_tools (end)

	\section{Prelab} % (fold)
	\label{sec:prelab}
		\subsection*{Question 1} % (fold)
		\label{sub:question_1}
			\begin{itemize}
				\item[a)] $s_{roots} = \{-2,-3,-4\}$
				\item[b)] $s_{roots} = \{0,-1,-5,-6\}$
				\item[c)] $P_3 = s^4 + 13s^3 + 50s^2 + 56s + 24$\\
						  $P_4 = -s^4 -11s^3 -32s^2 -4s + 24$ \\
						  $P_5 = \text{A mystery :'(}$
			\end{itemize}
		% subsection question_1 (end)

		\subsection*{Question 2} % (fold)
		\label{sub:question_2}
			\begin{equation}
				\begin{array}{rcl}
					P_6 & = & (s+7)(s+8)(s+3)(s+5)(s+9)(s+10) \\
					& = & s^6 + 42s^5 +718s^4 + 6\,732s^3 + 30\,820s^2 + 76\,530s + 7\,560
				\end{array}
			\end{equation}
		% subsection question_2 (end)

		\subsection*{Question 3a}
		\label{sub:question_3a}
			\begin{equation}
				\begin{array}{rcl}
					G_1(s) & = & \frac{20(s+2)(s+3)(s+6)(s+8)}{s(s+7)(s+9)(s+10)(s+15)}\\
					& = & \frac{20(s^3+5s^2+6s+6s^2+30s+36)(s+8)}{(s^3+7s^2+9s^2+63s)(s+10)(s+15)}\\
					& = & \frac{20(s^3+41s^2+36s+36)(s+8)}{(s^3+16s^2+63s)(s+10)(s+15)}\\
					& = & \frac{20s^4+380s^3+2480s^2+6480s+5760}{s^5+26s^4+223s^3+630s^2+15s^4+390s^3+3345s^2+9450s}\\
					& = & \frac{20s^4+380s^3+2480s^2+6480s+5760}{s^5+41s^4+613s^3+3975s^2+9450s}
				\end{array}
			\end{equation} 

		\subsection*{Question 3b}
		\label{sub:question_3b}
			Now using the Rationale Root Theorem to find the factors of the numerator and denominator, we find that $(s+2)$ is a factor of the numerator
			\begin{equation}
				\begin{array}{rcl}
				(s+2)(s^2+7s+12) & = & (s+2)(s+3)(s+4) \\
				\end{array}
			\end{equation}

	% section prelab (end)


	\section{Lab} % (fold)
	\label{sec:lab}
		\subsection*{Question 1} % (fold)
		\label{sub:question}
			This question was done by using the \texttt{roots} command. The polynomials were initialized as \par
			\texttt{P\_1 = [0,1,9,26,24];} \par
			\texttt{P\_2 = [1,12,41,30,0];}

			\noindent The resulting polynomials for this questions were obtained by \par
			\texttt{P\_3 = P\_1 + P\_2;} \par
			\texttt{P\_4 = P\_1 - P\_2;} \par
			\texttt{P\_5 = conv(P\_1, P\_2);} \par

			\noindent Running the \texttt{roots} command on the above polynomials, the roots were found to be
			\begin{equation*}
				\begin{array}{rcl}
					s_{P3} & = & \{-6.72,-4.81,-0.733 + j0.452, -0.733 - j0.452\} \\
					s_{P4} & = & \{-5.27 + j0.452,-5.27 - j0.452, -1.19, 0.722\} \\
					s_{P5} & = & \{0,-1,-2,-3,-4,-5,-6\}
				\end{array}
			\end{equation*}
		% subsection question (end)

		\subsection*{Question 2} % (fold)
		\label{sub:question_2}
			What???
		% subsection question_2 (end)

		\subsection*{Question 3} % (fold)
		\label{sub:question_3}
			\texttt{s = tf('s');} \par\noindent
			\texttt{G\_1 = 20*(s+2)*(s+3)*(s+6)*(s+8)/(s*(s+7)*(s+9)*(s+10)*(s+15));}

			\noindent Running the above commands gives the following result:
			\[
				G_1(s) = \frac{20 s^4 + 380 s^3 + 2480 s^2 + 6480 s + 5760}{s^5 + 41 s^4 + 613 s^3 + 3975 s^2 + 9450 s}
			\]

		% subsection question_3 (end)
	% section lab (end)

	\section{Postlab} % (fold)
	\label{sec:postlab}
		
	% section postlab (end)

	\section{References} % (fold)
	\label{sec:references}
		
	% section references (end)

\end{document}