\documentclass[12pt, a4paper]{article}
\usepackage{amsmath}
\usepackage{hyperref}

\hypersetup{colorlinks=true, linkcolor=black}

\title{Control Systems Practical 1}
\date{21 August 2018}
\author{Q. Kruger, 216008466 \and R. de Bruyn, 216969696}

\begin{document}
	\pagenumbering{gobble}
	\maketitle
	\newpage
	\pagenumbering{roman}
	\tableofcontents
	\newpage
	\pagenumbering{arabic}
	
	\section{Introduction} % (fold)
	\label{sec:introduction}
		This is an introductory practical to control systems computing using MATLAB/Octave. The purpose of this practical is to become familiar with the common functions used to analyze control systems.
	% section introduction (end)

	\section{Equipment and Tools} % (fold)
	\label{sec:equipment_and_tools}
		The only equipment used during this practical is the MATLAB/Octave scientific computing platforms.
	% section equipment_and_tools (end)

	\section{Prelab} % (fold)
	\label{sec:prelab}
		\subsection*{Question 1} % (fold)
		\label{sub:question_1}
			\begin{itemize}
				\item[a)] $s_{roots} = \{-2,-3,-4\}$
				\item[b)] $s_{roots} = \{0,-1,-5,-6\}$
				\item[c)] $P_3 = s^4 + 13s^3 + 50s^2 + 56s + 24$\\
						  $P_4 = -s^4 -11s^3 -32s^2 -4s + 24$ \\
						  $P_5 = \text{A mystery :'(}$
			\end{itemize}
		% subsection question_1 (end)

		\subsection*{Question 2} % (fold)
		\label{sub:question_2}
			\begin{equation}
				\begin{array}{rcl}
					P_6 & = & (s+7)(s+8)(s+3)(s+5)(s+9)(s+10) \\
					& = & s^6 + 42s^5 +718s^4 + 6\,732s^3 + 30\,820s^2 + 76\,530s + 7\,560
				\end{array}
			\end{equation}
		% subsection question_2 (end)
	% section prelab (end)

	\section{Lab} % (fold)
	\label{sec:lab}
		\subsection*{Question 1} % (fold)
		\label{sub:question}
			This question was done by using the \texttt{roots} command. The polynomials were initialized as \par
			\texttt{P\_1 = [0,1,9,26,24];} \par
			\texttt{P\_2 = [1,12,41,30,0];}

			\noindent The resulting polynomials for this questions were obtained by \par
			\texttt{P\_3 = P\_1 + P\_2;} \par
			\texttt{P\_4 = P\_1 - P\_2;} \par
			\texttt{P\_5 = conv(P\_1, P\_2);} \par

			\noindent Running the \texttt{roots} command on the above polynomials, the roots were found to be
			\begin{equation*}
				\begin{array}{rcl}
					s_{P3} & = & \{-6.72,-4.81,-0.733 + j0.452, -0.733 - j0.452\} \\
					s_{P4} & = & \{-5.27 + j0.452,-5.27 - j0.452, -1.19, 0.722\} \\
					s_{P5} & = & \{0,-1,-2,-3,-4,-5,-6\}
				\end{array}
			\end{equation*}
		% subsection question (end)

		\subsection*{Question 2} % (fold)
		\label{sub:question_2}
			\texttt{P\_6 = (tf('s')+7)*(tf('s')+8)*(tf('s')+3)*(tf('s')+5)*(tf('s')+9)*(tf('s')+10);} \par\noindent
			The result of the above command is
			\[
				P_6 = s^6 + 42 s^5 + 718 s^4 + 6372 s^3 + 3.082\times 10^{4} s^2 + 7.653\times10^4 s + 7.56\times 10^4
			\]
		% subsection question_2 (end)

		\subsection*{Question 3} % (fold)
		\label{sub:question_3}
			\texttt{s = tf('s');} \par\noindent
			\texttt{G\_1 = 20*(s+2)*(s+3)*(s+6)*(s+8)/(s*(s+7)*(s+9)*(s+10)*(s+15));}

			\noindent Running the above commands gives the following result:
			\[
				G_1(s) = \frac{20 s^4 + 380 s^3 + 2480 s^2 + 6480 s + 5760}{s^5 + 41 s^4 + 613 s^3 + 3975 s^2 + 9450 s}
			\]

		% subsection question_3 (end)
	% section lab (end)

	\section{Postlab} % (fold)
	\label{sec:postlab}
		Octave was able to do in a few minutes what would otherwise take a human half an hour to do by hand, and is a very convenient and intuitive way to do scientific computing. An interesting finding in the course of this practical, is that polynomials can be multiplied using convolution.
	% section postlab (end)

	\section{References} % (fold)
	\label{sec:references}
		
	% section references (end)

\end{document}