%        File: pwe2.tex
%     Created: Thu Apr 25 10:00  2019 S
% Last Change: Thu Apr 25 10:00  2019 S
%
\documentclass[a4paper, 12pt]{article}

\usepackage[]{amsmath, amssymb}
\usepackage{IEEEtrantools}
\usepackage[]{graphicx}
\usepackage{float}
\usepackage[toc, page]{appendix}
\usepackage[]{hyperref}

\title{Power Electronics: Designing and Building a Single Switch Flyback Converter}
\author{Ruan de Bruyn \and 216054484 \and Quintin Kruger \and 216008466}

\begin{document}

\pagenumbering{gobble}
\maketitle
\newpage
\pagenumbering{roman}
\tableofcontents
\listoffigures
\newpage
\pagenumbering{arabic}

\section{Introduction}

% endsection Introduction

\section{Design}

\subsection{Circuit Design}

The design specifications are given as such:
<+ADD DESIGN SPECS HERE+>

We used the recommended textbook the practical \cite{pressman} during the
design of this converter. Since we have a switching frequency of $f_s = 25kHz$,
that leaves us with a period of

\begin{IEEEeqnarray}{rCl}
	T = \frac{1}{f_s} \nonumber \\
	& = & 40 \mu s
	\label{eq:T}
\end{IEEEeqnarray}

For stability and dynamic response reasons, we decided to have this flyback
converter operate in discontinuous mode. Thus, we will have an ``on'' time,
$T_{on}$, which will represent our duty cycle. There needs to be a ``reset''
time as well, to allow the inductor to completely dispose of the energy in its
magnetic field. \cite{pressman} recommends the following constraint:

\begin{equation}
	T_{on} + T_r = 0.8T
	\label{eq:time_constraint}
\end{equation}

Equation \eqref{eq:time_constraint} ensures that there is enough dead time
between cycles for the inductor to reset. Then, with our maximum duty cycle
being 42.5\%, we have

\begin{equation}
	T_{on} = D_{\text{max}} T = 17 \mu s
	\label{eq:ton}
\end{equation}

\noindent and from equation \eqref{eq:time_constraint}

\begin{IEEEeqnarray}{rCl}
	T_r & = & 0.8(40 \mu s) - 17 \mu s \nonumber \\
	& = & 15 \mu s
	\label{eq:tr}
\end{IEEEeqnarray}

leaving us with a dead time of about $8 \mu s$ per cycle to ensure
discontinuous mode operation. Next, we need to calculate the turns ratio for
the coupled inductor. With the assumed efficiency of $\eta = 80\%$ and a
specified output power of $P_o = 20 W$, this means we would have input power
$P_i = 25 W$. From a 12V DC power supply, we would then expect to draw an
average current of approximately 2A on the primary side. This is important,
because when calculating our turns ratio, the voltage drop across the switch on
the primary side needs to be taken into account. According to the IRF510 Power
MOSFET datasheet, it has an $R_{DS, on}$ of $0.54\Omega$. Granted, this was
measured with a primary current of 3.4A according to the datasheet, but this
should suffice as an approximation for us. Then we have a forward voltage drop
over the switch of approximately $V_{DS} = (2)(0.54) \approx 1V$. 

Then, on the secondary side, we require an output voltage of $V_o = 18V$.
However, our fast switching 1N4148 diode has a forward voltage drop of $V_f =
1V$, which means that the coupled inductor has to step up voltage to compensate
for this drop. Taking all of this into account, together with the fact that the
integral of the volt-seconds over the inductor should equal zero, we have

\begin{equation}
	(V_i - V_{DS})T_{on} = \frac{N_s}{N_p}(V_o + V_f)T_r
	\label{eq:turnsratio}
\end{equation}

Letting $\rho$ denote our turns ratio, and substituting for our known values,
it follows from \eqref{eq:turnsratio} that

\begin{IEEEeqnarray}{rCl}
	\rho & = & \frac{V_i - V_{DS}T_{on}}{(V_o + V_f)T_r} \nonumber \\
	& = & \frac{(12 - 1)17 \times 10^{-6}}{(18 + 1)15 \times 10^{-6}} \nonumber \\
	& = & 0.656 \nonumber \\
	& \approx & 0.65
	\label{eq:rho}
\end{IEEEeqnarray}

The inductance of the primary side of the coupled inductor comes next. The
proof of the equation for the primary inductance is covered in Pressman
\cite{pressman}, and is beyond the scope of this report. We will, however,
alter the equation as a function of efficiency (\cite{pressman} assumes a fixed
efficiency), which yields

\begin{equation}
	V_o = V_{i} T_{on} \sqrt{\frac{R_o}{\tfrac{2}{\eta}T L_p}} \Rightarrow L_p = \frac{(V_i T_{on})^2}{\tfrac{2}{\eta} T P_o}
	\label{eq:lp}
\end{equation}

Substituting our values, it leaves us with a primary inductance of

\begin{IEEEeqnarray}{rCl}
	L_p & = & \frac{(12 \times 17 \times 10^{-6})^2}{\tfrac{2}{0.8}(40 \times 10^{-6})(20)} \nonumber \\
	& = & 2.08 \times 10^{-5} \nonumber \\
	& = & 20.8 \mu H
	\label{eq:primary_inductance}
\end{IEEEeqnarray}

Getting the secondary inductance can be done with a simple ratio:

\begin{IEEEeqnarray}{rCl}
	L_s & = & \rho^{-2}L_p \nonumber \\
	& = & 0.65^{-2}(2.08 \times 10^{-5}) \nonumber \\
	& = & 4.925 \times 10^{-5} \nonumber \\
	& = & 49.3 \mu H
	\label{eq:secondary_inductance}
\end{IEEEeqnarray}

% endsubsection Circuit Design

\subsection{Negative Feedback Control}

We decided to use a Raspberry Pi Model 3B+. <+PROVIDE DISCUSSION HERE+>

% endsubsection Negative Feedback Control

% endsection Design
\section{Construction}

% endsection Construction

\section{Results and Discussion}

% endsection Results and Discussion

\begin{thebibliography}
	\bibitem{pressman} A. Pressman, K. Billings, T. Morey, \textit{Switching Power Supply Design, Third Edition}, Chapter 4 --- Flyback Converter Topologies
\end{thebibliography}

\begin{appendices}
	\section{Negative Feedback Code}
	\texttt{\#include <wiringPi.h>} \\\noindent
	\texttt{\#include <stdio.h>} \\\noindent
	\texttt{\#include <stdlib.h>} \\\noindent
	\texttt{\#include <stdint.h>} \\\noindent
	\texttt{\#include <time.h>} \\\noindent
	\texttt{\#include <pthread.h>} \\\noindent
	\texttt{\#include <math.h>} \\\noindent
	\texttt{\#include <mcp3004.h>} \\\noindent
	\texttt{ \\\noindent}
	\texttt{\#define BASE 100} \\\noindent
	\texttt{\#define SPI\_CHAN 0} \\\noindent
	\texttt{ \\\noindent}
	\texttt{double period = 40;} \\\noindent
	\texttt{const double D\_MAX = 0.425;} \\\noindent
	\texttt{const double D\_MIN = 0.025;} \\\noindent
	\texttt{double duty\_cycle = D\_MIN;} \\\noindent
	\texttt{int voltage = 0;} \\\noindent
	\texttt{double normalized\_voltage = 0.0;} \\\noindent
	\texttt{double error = 0.0;} \\\noindent
	\texttt{int ref\_voltage = 18;} \\\noindent
	\texttt{const vref = 5;} \\\noindent
	\texttt{const double r\_factor = 21 * vref / 1024;} \\\noindent
	\texttt{double p = 0.0125;} \\\noindent
	\texttt{double p\_i = 0.003;} \\\noindent
	\texttt{ \\\noindent}
	\texttt{bool running = true;} \\\noindent
	\texttt{ \\\noindent}
	\texttt{void *run\_pwm()} \\\noindent
	\texttt{{ \\\noindent}
	\texttt{printf("Running pwm\n");} \\\noindent
	\texttt{ \\\noindent}
	\texttt{int fixed\_cycle = 0;} \\\noindent
	\texttt{int t\_on = 0;} \\\noindent
	\texttt{int t\_off = 0;} \\\noindent
	\texttt{int x;} \\\noindent
	\texttt{while(running){ \\\noindent}
	\texttt{fixed\_cycle = duty\_cycle;} \\\noindent
	\texttt{t\_on = round(period * fixed\_cycle);} \\\noindent
	\texttt{t\_off = round(period * (1 - fixed\_cycle));} \\\noindent
	\texttt{digitalWrite(1, LOW);} \\\noindent
	\texttt{delayMicroseconds(t\_on);} \\\noindent
	\texttt{digitalWrite(1, HIGH);} \\\noindent
	\texttt{delayMicroseconds(t\_off);} \\\noindent
	\texttt{}} \\\noindent
	\texttt{} \\\noindent
	\texttt{printf("We're done\n");} \\\noindent
	\texttt{digitalWrite(1, LOW);} \\\noindent
	\texttt{pinMode(1, 0);} \\\noindent
	\texttt{}} \\\noindent
	\texttt{ \\\noindent}
	\texttt{void *controller()} \\\noindent
	\texttt{{ \\\noindent}
	\texttt{const int size\_max = 10;} \\\noindent
	\texttt{double i\_values[size\_max] = {};} \\\noindent
	\texttt{int size = 0;} \\\noindent
	\texttt{double sum = 0;} \\\noindent
	\texttt{while(running)} \\\noindent
	\texttt{{ \\\noindent}
	\texttt{// run until cancel} \\\noindent
	\texttt{// get measurements} \\\noindent
	\texttt{// voltage = analogRead(BASE);} \\\noindent
	\texttt{// normalized\_voltage = voltage * r\_factor;} \\\noindent
	\texttt{// error = ref\_voltage - normalized\_voltage;} \\\noindent
	\texttt{ \\\noindent}
	\texttt{voltage = 0;} \\\noindent
	\texttt{normalized\_voltage = 0;} \\\noindent
	\texttt{error = 0;} \\\noindent
	\texttt{ \\\noindent}
	\texttt{// Add error value to integrator} \\\noindent
	\texttt{if(size < size\_max)} \\\noindent
	\texttt{size++;} \\\noindent
	\texttt{i\_values[size - 1] = error} \\\noindent
	\texttt{else { \\\noindent}
	\texttt{// shift old values} \\\noindent
	\texttt{int i;} \\\noindent
	\texttt{for(i = 0; i < size\_max - 1; i++)} \\\noindent
	\texttt{i\_values[i] = i\_values[i + 1];} \\\noindent
	\texttt{i\_values[size\_max - 1] = error;} \\\noindent
	\texttt{}} \\\noindent
	\texttt{ \\\noindent}
	\texttt{// sum values for integrator} \\\noindent
	\texttt{int i;} \\\noindent
	\texttt{sum = 0;} \\\noindent
	\texttt{for(i = 0; i < size, i++)} \\\noindent
	\texttt{sum += i\_values[i];} \\\noindent
	\texttt{ \\\noindent}
	\texttt{// adjust duty cycle accordingly} \\\noindent
	\texttt{duty\_cycle += p * error + p\_i * sum;} \\\noindent
	\texttt{ \\\noindent}
	\texttt{if(duty\_cycle > D\_MAX)} \\\noindent
	\texttt{duty\_cycle = D\_MAX;} \\\noindent
	\texttt{ \\\noindent}
	\texttt{if(duty\_cycle < D\_MIN)} \\\noindent
	\texttt{duty\_cycle = D\_MIN;} \\\noindent
	\texttt{ \\\noindent}
	\texttt{// Delay single microsecond} \\\noindent
	\texttt{delayMicroseconds(500);} \\\noindent
	\texttt{}} \\\noindent
	\texttt{}} \\\noindent
	\texttt{ \\\noindent}
	\texttt{int main(void) { \\\noindent}
	\texttt{// setup physical} \\\noindent
	\texttt{wiringPiSetup();} \\\noindent
	\texttt{mcp3004Setup(BASE, SPI\_CHAN);} \\\noindent
	\texttt{pinMode(1, OUTPUT); } \\\noindent
	\texttt{ \\\noindent}
	\texttt{pthread\_t thread\_1;} \\\noindent
	\texttt{pthread\_t thread\_2;} \\\noindent
	\texttt{pthread\_create(&thread\_1, NULL, run\_pwm, NULL);} \\\noindent
	\texttt{pthread\_create(&thread\_2, NULL, controller, NULL);} \\\noindent
	\texttt{ \\\noindent}
	\texttt{char* line = NULL;} \\\noindent
	\texttt{size\_t len = 0;} \\\noindent
	\texttt{ssize\_t read = 0;} \\\noindent
	\texttt{ \\\noindent}
	\texttt{while(running)} \\\noindent
	\texttt{{ \\\noindent}
	\texttt{while (read != -1) { \\\noindent}
	\texttt{puts("Enter q to quit");} \\\noindent
	\texttt{read = getline(&line, &len, stdin);} \\\noindent
	\texttt{printf("line = %s", line);} \\\noindent
	\texttt{printf(line);} \\\noindent
	\texttt{puts("");} \\\noindent
	\texttt{if (!strcmp(line, "q\n")) { \\\noindent}
	\texttt{running = false;} \\\noindent
	\texttt{break;} \\\noindent
	\texttt{}} \\\noindent
	\texttt{}} \\\noindent
	\texttt{}} \\\noindent
	\texttt{\\\noindent}
	\texttt{free(line);} \\\noindent
	\texttt{pthread\_join(thread\_1, NULL);} \\\noindent
	\texttt{pthread\_join(thread\_2, NULL);} \\\noindent
	\texttt{\\\noindent}
	\texttt{return 0;} \\\noindent
	\texttt{}} \\\noindent

\end{appendices}

\end{document}
