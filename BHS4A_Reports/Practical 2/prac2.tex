%        File: prac2.tex
%     Created: Sun Mar 03 05:00  2019 S
% Last Change: Sun Mar 03 05:00  2019 S
%
\documentclass[a4paper, 12pt]{article}
\usepackage[]{amsmath}
\usepackage{amssymb}
\usepackage{float}
\usepackage[]{graphicx}
\usepackage{subfig}
\usepackage{caption}

\title{Control Systems 4A Practical 2 Report}
\author{Ruan de Bruyn, 216054484 \and Quintin Kruger, 216008466}

\begin{document}

\pagenumbering{gobble}
\begin{titlepage}
  \maketitle
\end{titlepage}

\pagenumbering{roman}
\tableofcontents
\newpage
\pagenumbering{arabic}

\section{Question 1}
\subsection{Input-Output form}
From the schematic representation of a DC motor, we can use KVL in order to
find the input-output relationships of all the components in the circuit
expressed as voltages. This leaves us with

\begin{equation}
  \begin{array}{rcl}
    v & = & V_R + V_L + e \\
    & = & Ri + L\dot i + K \dot\theta
  \end{array}
  \label{eq:KVL}
\end{equation}

\noindent Note that \eqref{eq:KVL} is a better representation for us to solve
the input-output equations; the current through the DC Motor is also the
same current through the other components. From theory, we can relate the
back-emf of the motor to its rotational speed $\dot\theta$ with the constant
$K$, as well as expressing the current in the DC Motor with the following
equation:

\begin{equation}
  J\ddot\theta + b\dot\theta = Ki \Rightarrow i = \tfrac{J}{K} \ddot\theta + \tfrac{b}{K}\dot\theta
  \label{eq:dc_current}
\end{equation}

From \eqref{eq:KVL} and \eqref{eq:dc_current}, and letting $\dot\theta = \omega$, we get

\begin{equation}
  Kv = LJ\ddot\omega + (Lb + RJ)\dot\omega + (Rb + K^2)\omega
  \label{eq:final_voltage}
\end{equation}

From \eqref{eq:final_voltage}, if we want to express the equation in terms of
in input reference rotational speed $\omega_r$, then we can evaluate equation
\eqref{eq:final_voltage} by assuming that $\omega_r$ is steady state, and thus
that its derivatives are zero. We then come to an expression for voltage in
terms of $\omega_r$ as:

\begin{equation}
  \begin{array}{rcl}
    Kv & = & (Rb + K^2)\omega_r \\
    v & = & \frac{(Rb + K^2)\omega_r}{K}
  \end{array}
  \label{eq:ref_omega}
\end{equation}

Finally, substituting equation \eqref{eq:ref_omega} into
\eqref{eq:final_voltage}, we get our final input-output equation as

\begin{equation}
  (Rb + K^2)\omega_r = LJ\ddot\omega + (Lb + RJ)\dot\omega + (Rb + K^2)\omega
  \label{eq:final_ref_omega}
\end{equation}

\subsection{State-variable Matrix form}
In order to represent our system in state-variable matrix form, we choose to
model it with parameters $\omega$ and $i$. For the first equation of the
state-variable representation, we require $\dot\omega$ and $\dot i$. We get the
former by rearranging \eqref{eq:dc_current}, taking $\dot\theta = \omega$,
which yields

\begin{equation}
  \dot\omega = \frac{K}{J} i - \frac{b}{J} \omega
  \label{eq:ss_omega_dot}
\end{equation}

Similarly, we rearrange \eqref{eq:KVL}, and substitute \eqref{eq:ref_omega} to
get

\begin{equation}
  \begin{array}{rcl}
    Ri + L\dot i + K\omega & = & v \\
    Ri + L\dot i + K\omega & = & \frac{(Rb + K^2)}{K}\omega_r \\
    \dot i & = & \frac{(Rb + K^2)}{KL}\omega_r - \frac{R}{L} i - \frac{K}{L}\omega \\
  \end{array}
  \label{eq:ss_i_dot}
\end{equation}

Now we can get the state-variable equations as

\begin{equation}
  \left[
  \begin{array}{c}
    \dot \omega \\
    \dot i
  \end{array}
  \right]
  =
  \left[
  \begin{array}{cc}
    -\frac{b}{J} & \frac{K}{J} \\
    -\frac{K}{L} & -\frac{R}{L}
  \end{array}
  \right]
  \left[
  \begin{array}{c}
    \omega \\
    i
  \end{array}
  \right]
  +
  \left[
  \begin{array}{c}
    0 \\
    \frac{Rb + K^2}{LK}
  \end{array}
  \right]
  \left[
  \begin{array}{c}
    \omega_r
  \end{array}
  \right]
  \label{eq:ss_eq1}
\end{equation}

\begin{equation}
  y =
  \left[
  \begin{array}{cc}
    1 & 0
  \end{array}
  \right]
  \left[
  \begin{array}{c}
    \omega \\
    i
  \end{array}
  \right]
  +
  \left[
  \begin{array}{c}
    0
  \end{array}
  \right]
  \left[
  \begin{array}{c}
    \omega_r
  \end{array}
  \right]
  \label{eq:ss_eq2}
\end{equation}

\subsection{Transfer Function form}

From equation \eqref{eq:final_ref_omega}, we can derive the transfer function
of the system in terms of reference angular velocity by taking the Laplace
Transform:

\begin{equation}
  \begin{array}{rcl}
    \mathcal{L}[(Rb + K^2)\omega_r] & = & \mathcal{L}[LJ\ddot\omega + (Lb + RJ)\dot\omega + (Rb + K^2)\omega] \\
    (Rb + K^2)\Omega_r & = & [LJs^2 + (Lb + RJ)s + (Rb + K^2)]\Omega \\
  \end{array}
  \label{eq:ts_initial}
\end{equation}

From equation \eqref{eq:ts_initial}, we can get the transfer function as

\begin{equation}
  T(s) = \frac{\Omega}{\Omega_r} = \frac{Rb + K^2}{LJs^2 + (Lb + RJ)s + Rb + K^2}
  \label{eq:ts_final}
\end{equation}

\subsection{Block Form}

Why would you even make us do this on an actual report?



\section{Question 2}
In order to have position control for a motor as well, we simply add an
additional state variable, $\theta$, and then go back to expressing all the
equations in the previous section in terms of $\theta$, instead of $\omega$. In
Question 1, we used the substitution $\omega = \dot\theta$ because there were
no terms of the equation that actually contained $\theta$ as is. So, for
brevity, we omit the derivation, as it is virtually the same as for Question 1,
and alter our existing state-variable equations to accommodate $\theta$, and
applying $\dot\theta = \omega$ to get

\begin{equation}
  \left[
  \begin{array}{c}
    \dot \theta \\
    \dot \omega \\
    \dot i
  \end{array}
  \right]
  =
  \left[
  \begin{array}{ccc}
    0 & 1 & 0 \\
    0 & -\frac{b}{J} & \frac{K}{J} \\
    0 & -\frac{K}{L} & -\frac{R}{L}
  \end{array}
  \right]
  \left[
  \begin{array}{c}
    \theta \\
    \omega \\
    i
  \end{array}
  \right]
  +
  \left[
  \begin{array}{c}
    0 \\
    0 \\
    \frac{Rb + K^2}{LK}
  \end{array}
  \right]
  \left[
  \begin{array}{c}
    \omega_r
  \end{array}
  \right]
  \label{eq:ss_position_eq1}
\end{equation}

\begin{equation}
  y =
  \left[
  \begin{array}{rcl}
	  y_1 \\
	  y_2
  \end{array}
  \right]
  =
  \left[
  \begin{array}{ccc}
    1 & 0 & 0 \\
    0 & 1 & 0
  \end{array}
  \right]
  \left[
  \begin{array}{c}
    \theta \\
    \omega \\
    i
  \end{array}
  \right]
  +
  \left[
  \begin{array}{c}
    0 \\
    0
  \end{array}
  \right]
  \left[
  \begin{array}{c}
    \omega_r
  \end{array}
  \right]
  \label{eq:ss_position_eq2}
\end{equation}

\section{Question 3}

As given in the practical, the equation for current is slightly modified by
letting $\omega_r$ be evaluated as an error instead, due to the negative
feedback loop, thus replacing $\omega_r$ with $K_p[\omega_r(t) - \omega(t)]$.
The current equation then simply becomes

\begin{equation}
	\dot i = -\left[ \frac{K}{L} + \frac{K_P(Rb + K^2)}{LK} \right]\omega - \frac{R}{l}i + \frac{K_P(Rb + K^2)}{LK}\omega_r
	\label{eq:3_current}
\end{equation}

We can modify equation \eqref{eq:ss_position_eq1} to be

\begin{equation}
  \left[
  \begin{array}{c}
    \dot \theta \\
    \dot \omega \\
    \dot i
  \end{array}
  \right]
  =
  \left[
  \begin{array}{ccc}
    0 & 1 & 0 \\
    0 & -\frac{b}{J} & \frac{K}{J} \\
    0 & -\left[ \frac{K}{L} + \frac{K_P(Rb + K^2)}{LK} \right] & -\frac{R}{L}
  \end{array}
  \right]
  \left[
  \begin{array}{c}
    \theta \\
    \omega \\
    i
  \end{array}
  \right]
  +
  \left[
  \begin{array}{c}
    0 \\
    0 \\
    \frac{K_P(Rb + K^2)}{LK}
  \end{array}
  \right]
  \left[
  \begin{array}{c}
    \omega_r
  \end{array}
  \right]
  \label{eq:3_ss_position}
\end{equation}

The output system is exactly the same as in equation \eqref{eq:ss_position_eq2}.

\end{document}

