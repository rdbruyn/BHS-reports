\documentclass[a4paper, 12pt]{article}

\usepackage{IEEEtrantools}
\usepackage[]{amsmath}
\usepackage{amssymb}
\usepackage{float}
\usepackage[]{graphicx}
\usepackage{subfig}
\usepackage{caption}


\title{Control Systems 4A Practical 2 Report}
\author{Ruan de Bruyn, 216054484 \and Quintin Kruger, 216008466}

\begin{document}

\pagenumbering{gobble}
\begin{titlepage}
  \maketitle
\end{titlepage}

\pagenumbering{roman}
\tableofcontents
\newpage
\pagenumbering{arabic}

\section{Question 1}

<+DO THINGS HERE+>

% endsection Question 1

\section{Question 2}

In this question, we consider the following circuit:

<+IMAGE OF CIRCUIT HERE+>

\subsection{State-space Representation}

In order to find the general state-space equations of this circuit, we start
with a few preliminary equations.

\begin{equation}
  i_{C1} = C_1 \dot{V_{C1}}
  \label{eq:ic1}
\end{equation}

\begin{equation}
  i_{C2} = C_2 \dot{V_{C2}}
  \label{eq:ic2}
\end{equation}

Then, we use Kirchoff's current law to get the following equation:

\begin{IEEEeqnarray}{rCl}
  i_1 & = & i_{C1} + i_{C2} \nonumber \\
  & = & C_1 \dot{V_{C1}} + C_2 \dot{V_{C2}}
  \label{eq:kcl}
\end{IEEEeqnarray}

Using Kirchoff's Voltage law at the right-most loop, and substituting from
\eqref{eq:kcl}, we get

\begin{IEEEeqnarray}{rCl}
  V_{C1} & = & V_{R2} + V_{C2} \nonumber \\
  & = & i_{C2}R_2 + V_{C2} \nonumber \\
  & = & R_2 C_2 \dot{V}_{C2} + V_{C2} \nonumber \\
  R_2 C_2 \dot{V}_{C2} & = & V_{C1} - V_{C2} \nonumber \\
  \dot{V}_{C2} & = & \frac{1}{R_2 C_2} V_{C1} - \frac{1}{R_2 C_2} V_{C2}
  \label{eq:kvl2}
\end{IEEEeqnarray}

Then, using the results from \eqref{eq:kcl} and \eqref{eq:kvl2}, and performing
Kirchoff's Voltage Law at the loop with the input voltage, we get

\begin{IEEEeqnarray}{rCl}
  V_i & = & V_{R1} + V_{C1} \nonumber \\
  & = & i_1 R_1 + V_{C1} \nonumber \\
  & = & R_1 \left[ C_1 \dot{V}_{C1} + C_2 \dot{V}_{C2} \right] + V_{C1} \nonumber \\
  & = & R_1 C_1 \dot{V}_{C1} + R_1 C_2 \dot{V}_{C2} + V_{C1} \nonumber \\
  & = & R_1 C_1 \dot{V}_{C1} + R_1 C_2 \left( \frac{1}{R_2 C_2} V_{C1} - \frac{1}{R_2 C_2} V_{C2} \right) + V_{C1} \nonumber \\
  R_1 C_1 \dot{V}_{C1} & = & - \left( \frac{R_1 + R_2}{R_2} \right) V_{C1} +  \frac{R_1}{R_2} V_{C2} + V_i \nonumber \\
  \dot{V}_{C1} & = & -\frac{R_1 + R_2}{R_1 R_2 C_1} V_{C1} + \frac{1}{R_2 C_2} V_{C2} + \frac{1}{R_1 C_1} V_i
\end{IEEEeqnarray}

This leaves us with the following state-space matrices:

\begin{equation}
  A = \left[
  \begin{array}{cc}
    -\frac{R_1 + R_2}{R_1 R_2 C_1} & \frac{1}{R_2 C_2} \\
    \frac{1}{R_2 C_2} & -\frac{1}{R_2 C_2}
  \end{array}
  \right]
  \label{eq:ss_A}
\end{equation}

\begin{equation}
  B = \left[
  \begin{array}{c}
    \frac{1}{R_1 C_1} \\
    0
  \end{array}
  \right]
  \label{eq:ss_B}
\end{equation}

Assuming that we take our output across capacitor $C_2$, we have

\begin{equation}
  C = \left[
  \begin{array}{c c}
    0 & 1
  \end{array}
  \right]
  \label{eq:ss_C}
\end{equation}

and finally,

\begin{equation}
  D = \left[ 0 \right]
  \label{eq:ss_D}
\end{equation}

\subsection{Controllability and Observability}

Let us take the following values for our components: $R_1 = R_2 = 50\omega$,
$C_1 = 3.9 \mu F$, $C_2 = 4.7 \mu F$. Then we have

\begin{equation}
  A = \left[
  \begin{array}{cc}
    -1.03 \times 10^4 & 5.13 \times 10^3 \\
    4.26 \times 10^3 & -4.26 \times 10^3
  \end{array}
  \right]
  \label{eq:q_A}
\end{equation}

\noindent and

\begin{equation}
  B = \left[
  \begin{array}{c}
    5.13 \times 10^3 \\
    0
  \end{array}
  \right]
  \label{eq:q_B}
\end{equation}

Then we have controllability matrix

\begin{equation}
  C_M = \left[
  \begin{array}{c c}
    0 & 1 \\
    4.26 \times 10^3 & 4.26 \times 10^3
  \end{array}
  \right]
  \label{eq:q_cm}
\end{equation}

and observability matrix

\begin{equation}
  O_M = \left[
  \begin{array}{c c}
    5.13 \times 10^3 & -5.26 \times 10^7 \\
    0 & 2.18 \times 10^7
  \end{array}
  \right]
  \label{eq:q_om}
\end{equation}

Now, it is easy to see that both of these matrices have a non-zero determinant,
and therefore that they are invertible. This leads us to conclude that the
system is, in fact, controllable and observable.

\subsection{Transfer Function}

If we want to get the transfer function of our system, we can use our
previously obtained matrices, and use the equation

\begin{equation}
  T(s) = C\left( sI - A \right) B + D
  \label{eq:2_tf}
\end{equation}

We omit the specific steps for brevity, and obtain the expression

\begin{IEEEeqnarray}{rCl}
  T(s) & = & \frac{C \text{adj}\left( sI - A \right) B}{\text{det}\left( sI - A \right)} \nonumber \\
  & = & \frac{2.19 \times 10^7}{s^2 + 6.04 \times 10^3 s - 2.2 \times 10^7}
  \label{eq:2_tf_answer}
\end{IEEEeqnarray}

% endsection Controllability and Observability
% endsection Question 2

\section{Question 1} % (fold)
\label{sec:question_1}
Using the student number 216008466 gives the masses for this question to be $M = 2.8$kg, $m = 0.16$kg and the distance betweent the rod and the mass to be $3.5$

\subsection{Implementing the Linear State-space Model} % (fold)
\label{sub:implementing_the_linear_state_space_model}
The system implemented  using the student number as previously stated gives the state-space model as follows 
\begin{equation}
  A = \left[
  \begin{array}{c c c c}
   	0 & 1 & 0 & 0 \\
   	0.302 &  0 & 0 & 0\\
   	0 & 0 & 0 & 1\\
   	-0.56 &  0 &  0 &  0\\
  \end{array}
  \right]
  \label{eq:a_q1}
\end{equation}

\begin{equation}
  B = \left[
  \begin{array}{c}
   0 \\-\frac{5}{59}\\ 0\\\frac{1}{2.8}
  \end{array}
  \right]
  \label{eq:b_q1}
\end{equation}

\begin{equation}
  C = \left[
  \begin{array}{c c c c}
    1 & 0 & 0 & 0\\
    0 & 0 & 1 & 0\\
  \end{array}
  \right]
  \label{eq:c_q1}
\end{equation}


% subsection implementing_the_linear_state_space_model (end)

\subsection{Full State Feedback System} % (fold)
\label{sub:full_state_feedback_system}
A series of steps need to followed in order to realize control for this system, the procedure follows
\begin{enumerate}
	\item Finding the control law\\
	This step establishes pole locations of the closed-loop system yielding the desired dynamic response of the system
	\item Introducing the reference input with full state feedback
	\item something 
	\item something 
\end{enumerate}

To implement step 1 we must find a matrix $K$ describing feedback of the system containing 4 constants for the forth order system at hand. We find the constants by findign the values of $k_i$ in $K$ by comparing the equation $det[s\textbf{I}-(\textbf{A}-\textbf{BK})] = 0$ (control characteristic equation) with the eqaution of system poles $det[s\textbf{I}-\textbf{A}] = 0$ (system characteristic equation)


% subsection full_state_feedback_system (end)

% section question_1 (end)

\end{document}
